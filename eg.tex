\documentclass{article}

\usepackage[backend=biber,authordate]{biblatex-chicago}
\usepackage{graphicx}
\usepackage[english]{babel}
\usepackage{rotating}

\title{The growing-up of a World War II housing project: Erie Gardens, Port
Clinton, Ohio}
\author{Gabe DeFreitas}
\date{\today}

\addbibresource{ref.bib}
\frenchspacing

\begin{document}

\maketitle
\tableofcontents

\section{Introduction}

Here we consider the history of community formation in the Erie Gardens
subdivision of Port Clinton, Ohio, a city of population about six thousand
located halfway between Toledo and Cleveland on the coast of Lake Erie. Erie
Gardens was built by the US federal government around the start of World War II
to accommodate an influx of workers at a local ordnance factory. After the war,
the neighborhood was incorporated into the surrounding fabric of Port Clinton,
gaining a slum reputation for its low-cost housing stock.

After situating Erie Gardens within the historical context of World War II
federal housing and prefabricated building techniques, we present the history
of community-building and revitalisation efforts in the neighbourhood, starting
from the 1960s. We argue from historical evidence that Erie Gardens became a
vital community, despite its artificial origins and socioeconomic challenges.
In fact, these very aspects arguably account for the strength of grassroots
activism in Erie Gardens to this day.

The appendix contains a historical map of Erie Gardens which shows the original
street names and some landmarks from the original 1940s development.

\section{Description}

\begin{figure}[h]
	\centering
	\includegraphics[scale=0.5]{images/eg_aerial.jpg}
	\caption{Aerial view of Erie Gardens \parencite{ott-auditor}}
\end{figure}

Erie Gardens is a subdivision of Port Clinton, Ohio built around 1941 to house
workers at the Erie Ordnance Depôt about one mile to the northwest of Port
Clinton at the present site of Erie Industrial Park. In the 2010 census, the
area had a population of roughly 518 residents \parencite{2010census} living in
about 220 housing units \parencite{mar85}. Erie Gardens is bounded by West
Third Street to the north, Fremont Road to the south, Wilson Avenue to the
east, and Port Clinton's municipal boundary to the west. Most of the housing
stock consists of prefabricated longhouses divided between two or three
apartment units. These houses are situated along narrow streets (some little
more than sidewalks) forming an irregular grid pattern.

Port Clinton housing advocate Linda Hartlaub commented about Erie Gardens in a
1988 article that ``people didn't want to live there. It was considered the
slum of Port Clinton'' \parencite{murray88}. Negative local attitudes toward
Erie Gardens derive from the low-cost nature of its housing and infrastructure
combined with an isolated position within Port Clinton. Despite the attitudes,
many Erie Gardens residents report a positive view of life in the neighborhood,
as childhood resident Terry Witter expressed to the \textit{Port Clinton News
Herald}: ``Maybe there was a stigma there, but I didn't feel any of that. It
was just a wonderful place to grow up'' \parencite{trusty18}. Erie Gardens'
unique position within Port Clinton prompted its tradition of grassroots
organisation, which dates from at least 1969. But first we must place Erie
Gardens' process of community formation in the historical context of of World
War II-era factory housing.

\section{World War II federal housing construction}

World War II rapidly returned the US economy to full production, ending the
Great Depression. This led, in turn, to massive housing shortages across the
country, wherever workers flocked to factory towns to participate in the war
effort and revived economy. As \textcite[9-10]{marks12} notes, ``the total
wartime migration between 1941 and 1945 exceeded 30 million people, or
one-fifth of the nation's population\ldots the urgent need for housing\ldots to
accommodate the large influx of factory workers resulted in the construction of
units of simplified design which incorporated a minimum amount and variation of
materials that could be assembled rapidly and cheaply''. This ``rationalised''
housing was composed of prefabricated sheets of materials like concrete,
plywood, composition board, gypsum, and Cemesto \parencite[19]{marks12}. The
resulting units were low-lying and horizontally oriented structures (see
\ref{fig:block}) with simple interior plans.

\begin{figure}[h]
	\centering
	\includegraphics[scale=0.4]{images/eg_house_aerial.jpg}
	\caption{Aerial of a typical Erie Gardens block \parencite{ott-auditor}}
	\label{fig:block}
\end{figure}

Albert Farwell Bemis, a builder and major figure in wartime housing, proposed
three tenets for the rational house: simplify by eliminating specialised spaces
in favour of open floor plans; ``streamline the construction process''; and
practice industrial management (Taylorism) to build economies of scale into the
supply chain \parencite[16]{marks12}. \textcite{marks12} argues that the
wartime housing shortage actually changed American atitudes toward the
rationalised house, due to the pressing need to accommodate workers on the
homefront. This suggests that wartime neighborhoods like Erie Gardens
anticipated the suburban housing boom of the 1950s, which would transform the
look and feel of the US landscape.  The efficient house forms of Erie Gardens
thus contained the kernel of postwar home ownership. (See \ref{fig:house})

Although cheap prefabricated housing suited the immediate needs of the war,
there was nevertheless concern that concentrated districts of low-cost housing
could develop into slums after wartime production subsided
\parencite[13]{marks12}. John Blandford, upon becoming National Housing
Administrator in 1943, reversed the original plan to dismantle all emergency
housing within two years after the war. Although this plan was intended to
avoid the development of slums, Blandford arranged a less wasteful program that
would assist postwar homeowners in upgrading the units with better materials
\parencite{marks12}.

Although many World War II projects, like Erie Gardens and nearby MacArthur
Park in Sandusky, Ohio, ultimately did develop as low-income districts,
Blandford's plan recognised that wartime housing had the capacity to grow into
vital and full-fledged neighbourhoods after the war. Indeed, following the end
of the war the \textit{Sandusky Register} announced that ``the Public Housing
Authority plans to dispose of the Erie Gardens Housing project\ldots several
methods are being considered for disposal of the project, one of which would
give the present tenants first chance to buy individual units''
\parencite{jun55}.

\begin{figure}[h]
	\centering
	\includegraphics[scale=0.53]{images/eg_longhouse2.jpg}
	\caption{Representative Erie Gardens prefab house \parencite{defreitas}}
	\label{fig:house}
\end{figure}

\section{Community building and revitalisation}

The substandard quality of Erie Gardens' housing stock was recognised by the
end of the 1960s, barely 20 years after the neighbourhood's construction in
1941. The \textit{Sandusky Register} reported on 14 March 1969 the formation of
the Community Action Commission (CAC) for Erie Gardens, sponsored by the Office of
Economic Opportunity in the context of the War on Poverty. The headline
announced ``Gripes Voiced In PC'' and marked the beginning of Erie Gardeners'
recognition of their unique challenges, shared destiny, and need for grassroots
advocacy \parencite{14mar69}. Residents at the meeting vented a long list of
complaints with living in the neighbourhood, including a lack of recreational
opportunities, poor maintenance of streets, poor housing maintenance by
absentee landlords, inadequate trash collection, discrimination, and inadequate
drainage systems.

The organising meeting of CAC inaugurated a tradition of grassroots
organisation in Erie Gardens that continues to the present day. Shortly after
the initial meeting, CAC garnered interest and held several meetings for the
establishment of a neighbour's credit union, citing ``numerous complaints that
loans are difficult to get for housing improvements for the Garden units''
\parencite{25apr69}. Promoters of the credit union were aware of the step they
were taking toward self-sufficiency and financial independence. The
\textit{Register} reported that ``the meeting was successful, not only because
of the enthusiasm it created, but because it was attended by some thirty
people. One member stressed the fact that they would be on their own, grouped
together in a common bond, helping one another'' \parencite{25apr69}. Around
the same time, CAC noted that they had successfully lobbied the Port Clinton
City School District to provide playground equipment at the Erie Gardens
kindergarten property (today the school bus depôt) and that CAC would sponsor
local adult education, allowing residents to work toward a high school degree
equivalent \parencite{11apr69}.

\begin{figure}[h]
	\centering
	\includegraphics[scale=0.08]{images/church.jpg}
	\caption{Community church \parencite{defreitas}}
\end{figure}

Yet the formation of CAC indicated that Erie Gardens, for better or worse, was
situated \textit{outside} the Port Clinton mainstream. Resident Sammie
Yarborough raised the issue at a 27 March 1969 meeting by stating: ``This is
all city business which you can take to City Councilmen. I thought the purpose
of the CAC was for community self help by helping each other. What about
education, job training, things the city cannot provide?'' Her insight is
relevant because the CAC actually took on two roles, providing local support of
the kind Yarborough describes, but serving as an advocacy group on behalf of
Erie Gardens with the local power centres like Port Clinton City Council and
Port Clinton City Schools. Conditions in Erie Gardens were inadequate
\textit{because} the city had neglected its young neighbourhood.  Even the
neighbourhood credit union was only a necessity because of lending
discrimination by area banks. ``The credit union, if it could be gotten off the
ground, might be the best way for community self help'', noted Lynn Nichols at
the same meeting. \parencite{28mar69}.

By the 1970s, although momentum for the credit union had fizzled out to no
avail, CAC's organising had placed Erie Gardens back on City Council's
political agenda. At this time, direct federal involvement in community
planning was waning and being replaced by the devolved system of Community
Development Block Grants (CDBG), which gave more discretion to local
communities. The stage was set for an institutional push at City Hall for the
health of a newly vocal Erie Gardens. The city government collaborated with
Wood-Sandusky-Ottawa-Seneca Community Action Commission (WSOS) to gain CDBG
funds on behalf of the neighborhood for housing and water line rehabilitation.

The \textit{Register} reported in September 1975 that Port Clinton had applied
for funding for street resurfacing and construction of a parking lot at
Superior Court \parencite{eaton75}. By June 1980, Mayor John Fritz was calling
for a city-sponsored report on needs in Erie Gardens that ``might eventually be
incorporated into a full-scale redevelopment plan'' \parencite{jun80}. Fritz
noted many of the issues which had aroused residents at the 1969 meeting,
namely water lines, sewer lines, deteriorating buildings, and limited parking.
The city government and WSOS actively pursued grants during the 1980s, finally
receiving a \$655,000 Small Cities block grant in summer 1984 after two
unsuccessful applications. An additional \$125,000 was tacked on that fall
\parencite{smith84}. The money was distributed to property owners through a
combination of direct grants and no-interest or low-interest loans, depending
on income, including protections for low-income tenants to ensure that they
would not be pushed out after landlords rehabilitated their buildings
\parencite{mar85}. After the program had been carried out, housing advocate
Linda Hartlaub noted in 1988 that people were already noticing the difference
of revitalisation efforts \parencite{murray88}.

Today Erie Gardens retains its distinct character and tradition of neighbourly
coöperation. In the late 2000s, the West End Conestoga group was founded by a
coalition of the Mental Health and Recovery Board, Ottawa County United Way,
City of Port Clinton, and Erie Gardens residents to promote development in the
neighbourhood with an emphasis on social well-being. The name ``West End''
reflects the group's effort to rebrand the area and remove the negative
connotations associated with the name ``Erie Gardens''. This name is also
featured in the West End Community Park, the descendant of CAC's 1969
playground equipment, which was extensively remodelled during the
administration of Mayor Vince Leone (2011--2015), including all-new basketball
courts.

\section{Historical GIS Methodology}

The map presented here (Figure \ref{fig:oldmap} depicts the original street
names and some community locations in Erie Gardens. The old street names are
found in Port Clinton ordinance 33-56 of September 25, 1956, around the time of
Erie Gardens' annexation by the city \parencite{pc-code}. The ordinance changed
the makeshift names of a federal housing project (``A Street'', ``B Street'',
``West Loop Service Drive'') into real placenames, drawing from categories such
as presidents (``Hoover Drive'', ``Polk Drive''), Great Lakes (``Erie Court'',
``Superior Court''), and US states (``Maryland Street'', ``Ohio Street''). Erie
Gardens is also notable in Port Clinton for its ``named sidewalks'' (see
\ref{fig:sidewalk}) between Portage and Harding Drives. These paths were
originally lettered as A Street, B Street, C Street, D Street, and E Street.
The ordinance gave them a homier touch, opting for Abbey Lane, Bataan Lane,
Concord Lane, Delaware Lane, and Edison Lane.

\begin{figure}[h]
	\centering
	\includegraphics[scale=0.53]{images/sidewalk.jpg}
	\caption{One of Erie Gardens' named sidewalks viewed from Portage Drive
	\parencite{defreitas}}
	\label{fig:sidewalk}
\end{figure}

Although the vast majority of Erie Gardens consists of residential units, we
depict some community structures on the south end of the neighbourhood,
directly off of Fremont Road. These buildings and lots were repurposed over
time. Community Lot A, the present site of West End Community Park, was
adjacent to the Erie Gardens Kindergarten building, which today serves as the
Port Clinton City School District's bus depôt. Community Lot B was the adjacent
land across Portage Drive to the east and contained the Erie Gardens
Administration Building. This was later repurposed into the administrative
building for the Port Clinton Board of Education \parencite{nov61}. It served
this purpose until the construction of Port Clinton's new middle school in
2012. The community lot names from the Erie Gardens' plat are found on the
Ottawa County Auditor's Office interactive map \parencite{ott-auditor}. Also
noted on my map parking area off Superior Court funded by the city in the 1970s
in response to resident complaints \parencite{eaton75}.

\begin{figure}[h]
	\centering
	\includegraphics[scale=0.45]{images/newmap.png}
	\caption{Modern street map of Erie Gardens (West End) \parencite{osm}}
\end{figure}

The base vector data and current street map was imported from OpenStreetMap
\parencite{osm}. I manually added the historical names, buildings, and areas
using QGIS \parencite{qgis}. The raster background comes from the ArcGIS tile
server.

\begin{sidewaysfigure}[h]
	\includegraphics[scale=0.47]{images/oldmap.png}
	\caption{Map of Erie Gardens circa World War II (own work)}
	\label{fig:oldmap}
\end{sidewaysfigure}

\section{Conclusion}

This paper places Erie Gardens in the historical context of federal wartime
housing and outlines its development from a housing project into a full
constituent neighbourhood of Port Clinton with a strong resident advocacy base.
Grassroots organising beginning in 1969 led to the increased awareness and
interest at City Hall in improving infrastructural conditions in Erie Gardens.
As noted, despite many achievements since that time, community building efforts
in Erie Gardens/West End continue to this day. Finally, the historical map of
Erie Gardens shows the layout and placenames from the wartime era. Suggestions
for further inquiry include a comprehensive survey of Erie Gardens residential
structures and an investigation into how socioeconomic conditions of Erie
Gardens have varied over time.

\printbibliography

\end{document}
