\documentclass{article}

\usepackage[backend=biber]{biblatex}
\usepackage[english]{babel}

\title{Erie Gardens Mapping}
\author{Gabe DeFreitas}
\date{\today}

\addbibresource{references.bib}
\frenchspacing

\begin{document}

\maketitle

\section{Introduction}

This paper pursues two historical mapping projects in the Erie Gardens
subdivision of Port Clinton, Ohio, a town of about 6000 permanent residents
located on the Lake Erie shoreline about halfway between Toledo and Cleveland.
The Erie Gardens neighbourhood is a World War II-era federal housing project
that was built to accommodate an influx of workers at a local ordnance works.
After the war, the area was incorporated into the larger community of Port
Clinton, although it soon gained a reputation as a slum due to the low-cost
nature of its housing stock.

The first part of this paper presents a historical map of Erie Gardens during
its time as a federal housing project, presenting contemporary street names
(which have since changed) and building locations drawn from municipal
ordinances and newspaper research. In the second part, we look historically at
three revitalisation efforts that took place in the neighbourhood in the 1960s,
1980s, and 2000s. We follow up with a GIS plotting of property values in 1991
and 2018 to determine what economic effects, if any, these revitalisations had
on Erie Gardens. 

\section{Neighbourhood description}

Erie Gardens is a []-acre subdivision of Port Clinton built around 1941 to
house workers at the Erie Ordnance Depot about one mile northwest at the
present site of Erie Industrial Park. In the 2010 census, the area had a
population of about 518 residents \parencite{2010census} living in about 220
housing units \parencite{3-85}. Erie Gardens is bounded by West Third Street to
the north, Fremont Road to the south, Wilson Avenue to the east, and Port
Clinton's municipal boundary to the west (Figure 1). Most of the housing stock
consists of prefabricated longhouses divided up into one or two apartment
units. The street grid is highly irregular and has streets named after minor
presidents, states, and Great Lakes. Its local reputation as a slum derives
from the low-cost nature of its housing combined with its isolated position
within Port Clinton's larger city fabric (Figure 2). Despite the reputation,
many Erie Gardens residents report a positive view of their neighborhood, as
Terry Witter expressed in the \textit{Port Clinton News Herald}: ``Maybe there
was a stigma there, but I didn't feel any of that. It was just a wonderful
place to grow up.'' \parencite{trusty18}

\printbibliography

\end{document}
